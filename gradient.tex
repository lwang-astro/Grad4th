%なんか日本語でコメントしとかないとvimが文字コード間違える
\documentclass[11pt]{jsarticle}
\usepackage{amssymb}
\usepackage{amsmath}
\usepackage{newtxmath}
\usepackage{newtxtext}
\usepackage[utf8]{inputenc}
\usepackage{bm}

\title{Gradient term for the 4th-order symplectic integrator}
\author{似鳥啓吾}

\begin{document}
\maketitle

For the 4th-order forward symplectic integrator, we need to evaluate the
gradient term,
\begin{equation}
\bm G_i = \frac{1}{m_i} \frac{\partial}{\partial \bm r_i}
\left[ \sum_{j=1}^N \bm F_j \cdot \bm F_j \right]
= \frac{2}{m_i} \sum_{j=1}^N \left[ \left( \frac{\partial}{\partial \bm r_i} \bm F_j \right) \cdot  \bm F_j \right].
\end{equation}
Here, $\bm F_i$ is force on particle $i$ and $\bm F_{ij}$ contribution from
particle $j$, i.e.,
\begin{equation}
\bm F_i = \sum_{j \neq i}^N \bm F_{ij}
= \sum_{j \neq i}^N \frac{G m_i m_j}{| \bm r_j - \bm r_i |^3}(\bm r_j - \bm r_i).
\end{equation}
Thus,
\begin{equation}
\frac{\partial}{\partial \bm r_i} \bm F_j 
= 
\frac{\partial}{\partial \bm r_i} \left[ \sum_{k \neq j}^N \bm F_{jk} \right]
=
\begin{cases}
\displaystyle \sum_{k \neq i}^N \frac{\partial}{\partial \bm r_i} \bm F_{ik} & (i = j) \\
\displaystyle \frac{\partial}{\partial \bm r_i} \bm F_{ji} = - \frac{\partial}{\partial \bm r_i} \bm F_{ij} & (i \neq j)
\end{cases}.
\end{equation}
The summation remains only in the diagonal term and disappears elsewhere.
\begin{equation}
\bm G_i = \frac{2}{m_i} \sum_{j \neq i}^N \left[ \frac{\partial}{\partial \bm r_i} \bm F_{ij} \right] \cdot \left(\bm F_i - \bm F_j\right)
\end{equation}
For the $N$-body system, gradient of mutual force in $3 \times 3$ matrix is given in,
\begin{equation}
\frac{\partial}{\partial \bm r_i} \frac{(\bm r_j - \bm r_i)}{| \bm r_j - \bm r_i |^3}
=
\frac{-I}{| \bm r_j - \bm r_i |^3}
+
\frac{3 (\bm r_j - \bm r_i) \otimes (\bm r_j - \bm r_i)}{| \bm r_j - \bm r_i |^5},
\end{equation}
where $I$ is a unit matrix.

Finally we have
\begin{equation}
\bm G_i = 2 G \sum_{j \neq i}^N 
  m_j \left[ \frac{(\bm F_j - \bm F_i)}{| \bm r_j - \bm r_i |^3} - \frac{3 (\bm r_j - \bm r_i) \cdot (\bm F_j - \bm F_i)}{| \bm r_j - \bm r_i |^5} (\bm r_j - \bm r_i) \right].
\end{equation}
One can just replace the velocity by the force in the jerk formula to compute it.
Note that $\bm G_i h^2$ has a dimension of force (acceleration times mass).

\section{Materials}
\[
\tilde{ \bm F}_i = \bm F_i + \frac{h^2}{48} \frac{1}{m_i} \frac{\partial}{\partial \bm r_i} \left| \bm F \right|^2
\]

\[
	K(\tfrac16 h) D(\tfrac12 h) \tilde K(\tfrac23 h) D(\tfrac12 h) K(\tfrac16 h)
\]

\[
\begin{aligned}
&
\left[
\begin{pmatrix}
\partial_1 \\ 
\partial_2 \\ 
\partial_3 \\ 
\end{pmatrix}
\begin{pmatrix}
F_1 & F_2 & F_3
\end{pmatrix}
\right]
\begin{pmatrix}
F_1 \\ F_2 \\ F_3
\end{pmatrix} \\
&=
\begin{pmatrix}
\partial_1 (F_{12} + F_{13}) & \partial_1 F_{21} & \partial_1 F_{31} \\
\partial_2 F_{12} & \partial_2 (F_{23} + F_{21} ) & \partial_2 F_{32} \\
\partial_3 F_{13} & \partial_3 F_{23}  & \partial_3 (F_{31} + F_{32}) \\
\end{pmatrix}
\begin{pmatrix}
F_1 \\ F_2 \\ F_3
\end{pmatrix} \\
&=
\begin{pmatrix}
(\partial_1 F_{12})(F_1 - F_2) + (\partial_1 F_{13})(F_1 - F_3) \\
(\partial_2 F_{23})(F_2 - F_3) + (\partial_2 F_{21})(F_2 - F_1) \\
(\partial_3 F_{31})(F_3 - F_1) + (\partial_3 F_{32})(F_3 - F_2) \\
\end{pmatrix}
\end{aligned}
\]

\[
	| \bm F_\text{hard} + \bm F_\text{soft}|^2 - | \bm F_\text{hard}|^2 = | \bm F_\text{soft}|^2 + 2 \bm F_\text{hard} \cdot \bm F_\text{soft}
\]

\section{P$^3$T} 
Poisson bracket is
\[
	\{A, B\} = 
	\frac{\partial A}{\partial q}\frac{\partial B}{\partial p}  -
	\frac{\partial B}{\partial q}\frac{\partial A}{\partial p}.
\]
We split the Hamiltonian into a hard part and a soft part,
\[
	H = \underbrace{(T + V_H)}_\text{hard} + \underbrace{V_S}_\text{soft}.
\]
Now,
\begin{align}
\{T+V_H, V_S\} =& \frac{p \cdot F_S}{m} \\
\{\{T+V_H, V_S\}, V_S\} =& \frac{F_S \cdot F_S}{m} \label{eq:p3terr} \\
\{V_S, T+V_H\} =& -\frac{p \cdot F_S}{m} \\
\{\{V_S, T+V_H\}, T+V_H\} 
 =& -\frac{1}{m} \{p \cdot F_S, T+V_H \} \nonumber \\
 =& -\frac{1}{m} \left( \frac1m \left[p \cdot \frac{\partial F_S}{\partial q}\right]p  + F_S \cdot F_H \right)
\end{align}
The leading error term is (\ref{eq:p3terr}).
\end{document}
